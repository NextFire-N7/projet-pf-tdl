\documentclass[headings=standardclasses,parskip=half]{scrartcl}

\usepackage[french]{babel}
\usepackage[margin=3cm]{geometry}
\usepackage{graphicx}
\usepackage[hidelinks]{hyperref}

\titlehead{
    \begin{center}
        \includegraphics[width=5cm]{n7.png}
    \end{center}
}
\subject{Projet de Programmation Fonctionnelle\\et de Traduction des Langages}
\title{Rapport}
\subtitle{}
\author{Enzo PETIT \and Nam VU}
\date{13 janvier 2022}
\publishers{ENSEEIHT – 2SN-A}


\begin{document}

\maketitle

\tableofcontents

\pagebreak

\section{Introduction}
Ce projet constitue en l'extension du compilateur RAT \(\to\) TAM développé
lors des scéances de Traduction des Langages.
L'extension réalisée rajoute au compilateur
les fonctionnalités suivantes:

\begin{itemize}
    \item Pointeurs
    \item Assignations d'affectation
    \item Types nommés
    \item Enregistrements
\end{itemize}

Plus de détails sur les implémentation de chacunes
de ces fonctionnalités peuvent être trouvés dans les sections
correspondantes du rapport.

Dans notre rendu du projet, tous les points demandés ont été réalisés
et sont donc, a priori, fonctionnels.

\section{Extensions du langage}

\subsection{Pointeurs}

\subsubsection*{Jugements de typage}

TODO

\subsubsection*{Evolution des AST}

Un nouveau type \texttt{affectable} représentant les affectables du
langage est défini par \texttt{Ident of string/TDS.info\_ast} et
\texttt{Deref of affectable} pour les déréférencements.

Le type \texttt{expression} contient de plus
\texttt{Adresse of string/TDS.info\_ast} (adresse d'une variable),
\texttt{Null} (pointeur null) et \texttt{New of typ} (nouveau pointeur
de type \texttt{typ}).

Le type \texttt{typ} comprend un \texttt{Pointeur of typ} représentant
les pointeurs. Etant récursif il permet d'enchaîner les pointeurs.

\subsubsection*{Implémentation}

Dans les différentes passes une analyse récursive des affectables
a été rajoutée.

Pour permettre l'affectation du pointeur \texttt{null},
l'idée était de le considérer \texttt{Pointeur Undefined}
et d'autoriser les affectations/déclarations entre ce type
et n'importe quel autre pointeur.

La principale difficulté dans les pointeurs fût au niveau de
la génération de code : il faut récursivement déréférencer avec des
\texttt{LOADI} jusqu'à arriver au "niveau de déréférencement" voulu
et là faire un \texttt{STOREI/LOADI} de la taille de ce qu'on pointe.

\subsection{Assignation d'addition}

\subsubsection*{Jugements de typage}

TODO

\subsubsection*{Evolution des AST}

TODO

\subsubsection*{Implémentation}

L'addition-affectation de pointeurs est court-circuitée
(comparé à effectuer l'addition dans un premier temps
puis l'affectation), pour descendre plus rapidement les chaines
de pointeurs.

Ainsi au lieu de charger la variable pointée,
on garde également le dernier pointeur (qui pointe sur la variable)
en mémoire, on effectue le calcul, et on réutilise
le pointeur sauvegardé pour réaliser l'affectation.

\subsection{Types nommés}

\subsubsection*{Jugements de typage}

TODO

\subsubsection*{Evolution des AST}

Seul l'\texttt{AstSyntax} a été modifié :

\begin{itemize}
    \item Un type \texttt{typedef} a été déclaré avec pour seul
          constructeur \texttt{TypedefGlobal of string * typ} et
          représente les typedefs globaux.
    \item Un constructeur \texttt{TypedefLocal of string * typ} a
          été rajouté à \texttt{instruction} pour les typedefs
          déclarés dans un bloc.
    \item Le constructeur du type \texttt{programme} évolue en
          \texttt{Programme of typedef list * fonction list * bloc}
          pour inclure les typedefs globaux.
\end{itemize}

Le type \texttt{typ} s'est vu rajouter un constructeur
\texttt{NamedTyp of string} qui représentent les types nommés.

\subsubsection*{Implémentation}

Le lexer et le parser ont été adaptés pour respecter la nouvelle
grammaire (identifiants de type, mot clé \texttt{typedef}).

Pour la suite la stratégie adoptée a été de "mettre à plat" tous les
types utilisés dans la passe TDS afin de n'avoir plus que des types
primitifs dans les passes suivantes.

Pour cela, une \texttt{analyse\_tds\_type} a été rajoutée dans
\texttt{PasseTdsRat} et a pour but de donner un équivalent du type passé
en entrée en type primitif.
Il ne vérifie aucun typage, juste si un type nommé utilisé a bien été
déclaré en amont.

Suite à ça, les passes suivante n'ont pas eu à être modifiées, tout
est comme si il n'y avait pas de types nommés dans le langage.

\subsection{Enregistrements}

\subsubsection*{Jugements de typage}

TODO

\subsubsection*{Evolution des AST}

TODO

\subsubsection*{Implémentation}

TODO

\section{Conclusion}

\end{document}
