\documentclass[headings=standardclasses,parskip=half]{scrartcl}

\usepackage[french]{babel}
\usepackage[margin=3cm]{geometry}
\usepackage{graphicx}
\usepackage[hidelinks]{hyperref}

\titlehead{
    \begin{center}
        \includegraphics[width=5cm]{n7.png}
    \end{center}
}
\subject{Projet de Programmation Fonctionnelle\\et de Traduction des Langages}
\title{Rapport}
\subtitle{}
\author{Enzo PETIT \and Nam VU}
\date{13 janvier 2022}
\publishers{ENSEEIHT – 2SN-A}


\begin{document}

\maketitle

\tableofcontents

\pagebreak

\section{Introduction}
Ce projet constitue en l'extension du compilateur RAT -> TAM développé lors des scéances de Traduction des Langages.
L'extension réalisée rajoute au compilateur les fonctionnalités suivantes:

\begin{itemize}
    \item Pointeurs
    \item Assignations d'affectation
    \item Types nommés
    \item Enregistrements
\end{itemize}

Plus de détails sur les implémentation de chacunes de ces fonctionnalités peuvent être trouvés dans les sections correspondantes du rapport.

Dans notre rendu du projet, tous les points demandés ont été réalisés et sont donc, a priori, fonctionnels.

\section{Extensions du langage}

On y détaillera toutes les modifications apportées

\subsection{Pointeurs}

\subsubsection*{Jugements de typage}

TODO

\subsubsection*{Evolution des AST}

Un nouveau type \texttt{affectable} représentant les affectables du
langage est défini par \texttt{Ident of string/TDS.info\_ast} et
\texttt{Deref of affectable} pour les déréférencements.

Le type \texttt{expression} contient de plus
\texttt{Adresse of string/TDS.info\_ast} (adresse d'une variable),
\texttt{Null} (pointeur null) et \texttt{New of typ} (nouveau pointeur
de type \texttt{typ}).

Le type \texttt{typ} comprend un \texttt{Pointeur of typ} représentant
les pointeurs. Etant récursif il permet d'enchaîner les pointeurs.

\subsubsection*{Implémentation}

Dans les différentes passes une analyse récursive des affectables
a été rajoutée.

Pour permettre l'affectation du pointeur \texttt{null},
l'idée était de le considérer de type \texttt{Pointeur Undefined}
et d'autoriser les affectations/déclarations entre ce type
et n'importe quel autre pointeur.

La principale difficulté dans les pointeurs fût au niveau de
la génération de code : il faut récursivement déréférencer avec des
\texttt{LOADI} jusqu'à arriver au "niveau de déréférencement" voulu
et là faire un \texttt{STOREI/LOADI} de la taille de ce qu'on pointe.

\subsection{Assignation d'addition}

\subsubsection*{Jugements de typage}

TODO

\subsubsection*{Evolution des AST}

TODO

\subsubsection*{Implémentation}

L'addition-affectation de pointeurs est court-circuitée
(comparé à effectuer l'addition dans un premier temps
puis l'affectation), pour descendre plus rapidement les chaines
de pointeurs.

Ainsi au lieu de charger la variable pointée,
on garde également le dernier pointeur (qui pointe sur la variable)
en mémoire, on effectue le calcul, et on réutilise
le pointeur sauvegardé pour réaliser l'affectation.

\subsection{Types nommés}

\subsubsection*{Jugements de typage}

TODO

\subsubsection*{Evolution des AST}

TODO

\subsubsection*{Implémentation}

TODO

\subsection{Enregistrements}

\subsubsection*{Jugements de typage}

TODO

\subsubsection*{Evolution des AST}

TODO

\subsubsection*{Implémentation}

TODO

\section{Conclusion}

\end{document}
