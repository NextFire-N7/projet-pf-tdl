\documentclass[headings=standardclasses,parskip=half]{scrartcl}

\usepackage[french]{babel}
\usepackage[margin=3cm]{geometry}
\usepackage{graphicx}
\usepackage[hidelinks]{hyperref}

\titlehead{
    \begin{center}
        \includegraphics[width=5cm]{n7.png}
    \end{center}
}
\subject{
    Projet de Programmation Fonctionnelle\\
    et de Traduction des Langages
}
\title{Rapport}
\subtitle{}
\author{Enzo PETIT \and Nam VU}
\date{13 janvier 2022}
\publishers{ENSEEIHT – 2SN-A}


\begin{document}

\maketitle

\tableofcontents

\pagebreak

\section{Introduction}

Ce projet cherche à étendre les capacités du compilateur RAT \(\to\)
TAM développé lors des séances de TP/TD de Traduction des Langages.
Les extensions réalisées rajoutent au compilateur
les fonctionnalités suivantes:

\begin{itemize}
    \item Pointeurs
    \item Opérateur d'assignation d'addition
    \item Types nommés
    \item Enregistrements
\end{itemize}

Les détails d'implémentation des différentes extensions sont énumérés
dans la section suivante.

Dans notre rendu du projet, tous les points demandés ont été réalisés
et sont à priori fonctionnels, excepté le traitement bonus des structures
récursives.

Des tests ont par ailleurs été rédigés et passent avec avec succès.

\section{Extensions du langage}

\subsection{Pointeurs}

\subsubsection*{Jugements de typage}

\[\sigma \vdash null : Pointeur(Undefined)\]
\[\frac{\sigma \vdash T : \tau}{\sigma \vdash new\ T : Pointeur(\tau)}\]
\[\frac{\sigma \vdash id : \tau}{\sigma \vdash \&id : Pointeur(\tau)}\]
\[\frac{\sigma \vdash a : Pointeur(\tau)}{\sigma \vdash *a : \tau}\]

\subsubsection*{Evolution des AST}

Un nouveau type \texttt{affectable} représentant les affectables du
langage est défini par \texttt{Ident of string/TDS.info\_ast} et
\texttt{Deref of affectable} pour les déréférencements.

Le type \texttt{expression} contient de plus
\texttt{Adresse of string/TDS.info\_ast} (adresse d'une variable),
\texttt{Null} (pointeur null) et \texttt{New of typ} (nouveau pointeur
de type \texttt{typ}).

Le type \texttt{typ} comprend un \texttt{Pointeur of typ} représentant
les pointeurs. Etant récursif il permet d'enchaîner les pointeurs.

\subsubsection*{Implémentation}

Dans les différentes passes une analyse récursive des affectables
a été rajoutée.

Pour permettre l'affectation du pointeur \texttt{null},
on utilise le type \texttt{Pointeur Undefined}
et on a autorisé les affectations/déclarations entre ce type
et n'importe quel autre pointeur.

La principale difficulté dans les pointeurs fût au niveau de
la génération de code : il faut récursivement déréférencer avec des
\texttt{LOADI} jusqu'à arriver au \textit{niveau de déréférencement}
voulu et là faire un \texttt{STOREI/LOADI} de la taille de ce
qu'on pointe.

\subsection{Opérateur d'assignation d'addition}

\subsubsection*{Jugements de typage}

\[\frac{\sigma \vdash a : int \quad \sigma \vdash e : int}
    {(a,int)::\sigma \vdash a \mathrel{+}= e : void}\]
\[\frac{\sigma \vdash a : rat \quad \sigma \vdash e : rat}
    {(a,rat)::\sigma \vdash a \mathrel{+}= e : void}\]

\subsubsection*{Evolution des AST}

De manière similaire au traitement de l'affectation, on a tout d'abord
ajouté une nouvelle instruction \texttt{AddAff}, permettant de
représenter l'addition-affectation, dans les AST \texttt{Syntaxe} et
\texttt{TDS}.

Puis, de manière équivalent au traitement de l'addition, on gère la
surcharge dans la passe de typage grâce à deux nouvelles instructions
qui viennent remplacer la première en fonction du type de l'opération
(sur des rationnels ou sur des entiers).

\subsubsection*{Implémentation}

Pour des variables normales, l'implémentation de l'addition-affectation
ne rajoute pas de point particuliers comparé aux implémentations
respectives de l'addition et de l'affectation.

L'addition-affectation de pointeurs est court-circuitée
(comparé à effectuer l'addition dans un premier temps
puis l'affectation), pour descendre plus rapidement les chaines
de pointeurs.

Au lieu de charger uniquement la variable pointée,
on garde également le dernier pointeur (qui pointe sur la variable)
en mémoire, on effectue le calcul, et on réutilise
le pointeur sauvegardé pour réaliser l'affectation.

\subsection{Types nommés}

\subsubsection*{Jugements de typage}

\[\frac{\sigma \vdash Tid : \tau}
    {\sigma \vdash typedef\ Tid = \tau : void, [Tid,\tau]}\]

\subsubsection*{Evolution des AST}

Seul l'\texttt{AstSyntax} a été modifié :

\begin{itemize}
    \item Un type \texttt{typedef} a été déclaré avec pour seul
          constructeur \texttt{TypedefGlobal of string * typ} et
          représente les typedefs globaux.
    \item Un constructeur \texttt{TypedefLocal of string * typ} a
          été rajouté à \texttt{instruction} pour les typedefs
          déclarés dans un bloc (locaux).
    \item Le constructeur du type \texttt{programme} évolue en
          \texttt{Programme of typedef list * fonction list * bloc}
          pour inclure les typedefs globaux.
\end{itemize}

Le type \texttt{typ} s'est vu rajouter un constructeur
\texttt{NamedTyp of string} qui représente les types nommés.

\texttt{InfoTyp of string * typ} a été rajouté pour enregistrer
les types nommés dans la TDS.

\subsubsection*{Implémentation}

Le lexer et le parser ont été adaptés pour respecter la nouvelle
grammaire (identifiants de type, mot clé \texttt{typedef}).

Pour la suite, la stratégie adoptée a été de \textit{mettre à plat}
tous les types utilisés dans la passe TDS afin de n'avoir plus que
des types primitifs dans les passes suivantes.

Pour cela, une \texttt{analyse\_tds\_type} a été rajoutée dans
\texttt{PasseTdsRat} et a pour but de donner un équivalent du type
passé en entrée en type primitif.
Il ne vérifie aucun typage, juste si un type nommé utilisé a bien été
déclaré en amont.

Suite à ça, les passes suivante n'ont pas eu à être modifiées, tout
est comme si il n'y avait pas de types nommés dans le langage.

\subsection{Enregistrements}

\subsubsection*{Jugements de typage}

\[\frac{\sigma \vdash A : Struct([...,x_k : \tau,...])}
    {\sigma \vdash (A.x_k) : \tau}\]
\[\frac{\sigma \vdash E_1:\tau_1 \quad ... \quad\sigma \vdash E_n:\tau_n}
    {\sigma \vdash \{E_1 \space ...\space E_n \} :
        Struct([x_1: \tau_1,\space...,\space x_n:\tau_n])}\]

\subsubsection*{Evolution des AST}

Le type \texttt{affectable} est amendé d'un nouveau constructeur
\texttt{Acces of affectable * string/TDS.info\_ast} représentant les
accès à un champ d'un enregistrement.

Dans le type \texttt{expression}, \texttt{StructExpr of expression list}
représente les expressions de structures.

Le type \texttt{typ} comprend un nouveau constructeur
\texttt{Struct of (typ * string) list} représente le type enregistrement
(liste des champs).

Dans le type \texttt{info}, deux nouveaux constructeurs
\texttt{InfoStruct of string * typ * int * string * info\_ast list} et
\texttt{InfoAttr of string * typ * int}
ont été rajoutés pour désigner respectivement une structure et un champ.

\subsubsection*{Implémentation}

Sans rentrer dans les détails du code, dans la passe TDS on a rajouté
une nouvelle fonction \texttt{analyse\_tds\_struct}, utilisée dans
\texttt{analyse\_tds\_type} définie précédemment, qui se charge de
vérifier si l'utilisation de la structure est correcte (pas de double
déclaration, etc.). On y a aussi rajouté une fonction
\texttt{creer\_info\_ast} qui renvoie une \texttt{info\_ast}
d'\texttt{InfoVar} ou d'\texttt{InfoStruct} le cas échéant.
\texttt{analyse\_tds\_affectable} a aussi été modifié pour vérifier
que les accès sur les variables sont corrects (fonction auxiliaire
\texttt{analyse\_acces}).

Pour la passe type, on a dû adapter \texttt{est\_compatible}
pour les structures.

Dans la passe de placement, \texttt{analyse\_placement\_struct} se
charge de placer les attributs des structures, relativement à la base
de la structure.

Enfin, dans la passe de génération de code,
\texttt{generer\_code\_affectable} a été modifié afin de pouvoir
y forcer un \textit{offset} et un \textit{type} lors des accès
à l'attribut d'une structure.

\section{Conclusion}

La difficulté du projet résidait surtout dans la détermination des
bonnes structures de données et le choix des traitements à faire
lors des différentes passes pour les différentes extensions.
La programmation OCaml n'était pas réellement un obstacle
sur ce projet, les itérateurs et le filtrage aidant beaucoup au
final.

Le déboggage n'était néanmoins pas très aisé : pas de réelle possibilité
de lancer un test particuler avec des breakpoints ou de logger des
valeurs sur la sortie standard facilement.

Concernant les enregistrements, les contraintes sur les bonnes
définitions des structures, censées \textit{simplifier les analyses},
nous ont parues assez compliqué à vérifier en pratique, du moins dans
notre version du compilateur\dots

Il n'en reste pas moins qu'un projet qu'on a trouvé globalement
enrichissant et assez agréable à réaliser !\\

\textit{PS: Nous avons trouvé que l'extension
    \href{https://marketplace.visualstudio.com/items?itemName=ocamllabs.ocaml-platform}{OCaml Platform}
    pour Visual Studio Code s'averrait bien plus fonctionnelle que
    celle conseillée en TP\dots}

\end{document}
